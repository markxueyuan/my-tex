\documentclass{article}

\begin{document}

\section{macros}

\def\macro{This here is actually the replacement text.}
Executing it" `\macro'.

\def\macro#1{replacement with first argument=#1}
Invoking it: \macro{hello!}

\def\macro#1-#2.{replacement with arguments: `#1' and `#2'.}
Invoking it: \macro contagious-contiguous.
% macro

\def\macro(#1,#2,#3){I found arguments `#1',`#2' and `#3'.}
\macro (43,44,45)

------------------------------------------------
\def\macro{Replacement \textmacro text \count0=42 \the\count0.}
\message{Debug message: `\meaning\macro'}

\def\macro#1.{This #1 is weird.}
\message{Let's find out: `\meaning\macro pi.'}
\macro pi.
% It is refreshing to realize that, at the time of definition, Tex does not process the replacement text.
% A kind of laziness

\def\macroone{This is macro one}
\def\macrotwo{macro two contains \macroone.}
Now, I execute it: \macrotwo.

\def\macroone{Redefined macroone}
Now, I execute the second macro again: \macrotwo.

\def\macroone#1#2.{#1 and #2 are affinities.}
\def\macrotwo{Alice}
\def\macrothree{John}
Believe it or not, \macroone\macrotwo\macrothree.

\def\macroone abc{\macrotwo}
\def\macrotwo def{\macrothree}
\def\macrothree#1{Got `#1', hehe}
Now show me: \macroone abcdefg
%There are no new things under sun.


------------------------------------------

% expand after
\def\point(#1,#2){The coordinate is #1 and #2.}
\def\temp{(1,3)}
\expandafter\point\temp

\def\theimportantmacro#1{I got the pre-assembled argument `#1' here.}
\def\temp{xyz}
\theimportantmacro\temp
% It seems that we can acheive the same effect without using \expandafter

\def\tempmacro#1{hehe, #1 is a fool}
\tempmacro{hello}

\tempmacro {hello}

\def\tempmacro#1-#2{hehe, #1 is different from #2}
\tempmacro {Ali-ce}-{A-lice}

% All above discussions seem make the following complex unnecessary

\def\tempmacro#1{I got the pre-assembled argument `#1' here.}
\def\temp{xyz}
\expandafter\tempmacro\expandafter{\temp}

% the next token is expanded only once
\def\macroone{This is macro one \macrotwo}
\def\macrotwo{--2--}
\def\macrothree#1{\def\macrofour{4[#1]}}
\expandafter\macrothree\expandafter{\macroone}
So far, nothing has been typeset. But now: \macrofour
\message{We have macrofour = \meaning\macrofour}
% By far, nothing new under sun.

% the \edef command expands, which is different from \def.

\def\a{3}
\def\b{2\a}
\def\c{1\b}
\edef\d{value=\c}
\message{Macro `d' is defined to be \meaning\d}
\expandafter\def\expandafter\d\expandafter{\c}
\message{Macro`d' is defined to be `\meaning\d' using expandafter}
% This reminds us that \expandafter only expands once.

% any assignment operation cannot be expanded

\edef\d{\count0=42}
\message{Macro `d' is defined to be `\meaning\d'}
\the\count0 % This shouldn't be astonishing, right?

\the\count1
\the\dimen0

\def\a{1234}
\edef\d{advance\count0 by \a}
\message{Macro `d' is defined to be `\meaning\d'}

\edef\d{Invoke \noexpand\a another macro}
\message{Macro `d' is defined to be `\meaning\d'}

------------------------------------------------

% Token Registers

\def\token{abcdefg}
\def\a{123}
\def\b{456}

\toks0={A \token list \a \b \count0=42 will never be expanded}
\edef\d{\the\toks0 }
\message{Macro `d' is defined to be `\meaning\d'}% tokens never be expanded
\the\toks0

% About absortion, a specialized experiment

\def\c{1234}
\count0=\c % manually left whitespace does not work
345 absorbed into \the\count0


\toks0={abc}
\toks1={\v}
\edef\d{\the\toks0}
\d1

\toks0={abc}
\toks1={DEF}
The value is now \the\toks01

% assemble several macros in one

\def\piecea{\a{xyz}}
\def\pieceb{\count0=42}
\def\piecec{string \b}
\toks0=\expandafter{\piecea}
\toks1=\expandafter{\pieceb}
\toks2=\expandafter{\piecec}
\edef\d{I have \the\toks0 and \the\toks1 and \the\toks2}
\message{Macro `d' is defined to be `\meaning\d'}

%more macro definition commands

%let

\def\b{abc}
\edef\c{def}
\newtoks\d
\d={ghk}
\let\f=\b
\f

\let\f=\c
\f

\let\f=\d
\the\f

\let\f=a
\f

\let\f=abc
\f

%\gdef is a short cut for \global\def

%\xdef is a short cut for \global\edef

%\cs name

\def\macro{Content}
This here is normal usage: `\macro'.
This here uses csname: `\csname macro\endcsname% There shouldn't be a space between name and \endcsname
% cs means control sequence.

\expandafter\def\csname a01macro with.strange.chars\endcsname{Content}
I use a strange macro. Here it is: `\csname a01macro with.strange.chars\endcsname`

%\csname\endcsname will expand all the tokens between them, like \edef

% the fact that the tokens in between are expandable allows to use `indirect' macro names:

\def\macro{onetwothree}
\expandafter\def\csname macro\macro\endcsname{Content}
I have just defined \expandafter\string\csname macro\macro\endcsname

with replacement text `\csname macro\macro\endcsname'

% Due to the flexibility showed above, \csname is used to implement all(?) of the key-value packages available in TeX.

%\string

\def\macro{Content}
I have just defined `\string\macro' using `\string\def'

%At first, I hope to see inside of \def as follows:
%\expandafter\string\csname \def\endcsname
% which means that \def is not a normal macro

% tools used to debug

\tracingmacros=2

\tracingcommands=2

\tracingrestores=1





\end{document}